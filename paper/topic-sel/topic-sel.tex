\documentclass[12pt, letterpaper]{article}


% - package setup
\usepackage[doublespacing]{setspace}

% - bib setup
\usepackage[style=apa]{biblatex}
\addbibresource{../bibliography.bib}

% - title setup
\title{Literary Construction of Attica in Riot-Era Cultural Production}
\author{Crystal Mandal}

\begin{document}

\maketitle

\subsection*{Topic Selection}
I don't know anyone in prison. It's a question I've been asked 
when I talk about prison abolition in any way or mention that 
I'm interested in carceral politics. I wish I could say that 
I have a wide political interest and my interest in prison 
abolition is one of many injustices I feel strongly about, but 
that is unfortunately not the case. 

I am a music major. In 2021, renowned pianist, composer, and 
political activist Frederic Rzewski passed away, spurring a 
resurgence of interest in his music. It was in the aftermath 
of his death that I first heard and appreciated his music. 
Of particular interest to me was the distinctly American 
political voice in his music. Much of his music experiments 
with American musical form and style, especially contextualised 
in critique and/or conversation with the contemporaneous 
political atmosphere. Of distinguished interest to me is his 
couplet of pieces ``Coming Together" and ``Attica", written in 
the 1970's. Both pieces, written in the aftermath of the Attica 
Riots, paint an image of Attica not only as brutal and transformative 
but also distinctly as \textit{unnecessary}. This dethroning of 
incarceration as a necessary social structure is not something 
that I was aware of prior to engagement with Rzewski's music, 
the emotional reponse I had and still do have to his ouvre and 
those two pieces in general fully inspired me to do further 
reading and become invested.

The topic I'm most interested in researching is less about 
the actualities of incarceration and carceral policy, but 
more about the representation of American Prison Systems in 
contemporary conversation. The Attica Riots were remarkable, publicized, 
and revolutionary, occupying a great deal of contemporaneous public 
consciousness, but have since faded into obscurity. I wish, 
by analysing and relating different poems, letters, pieces of 
art and music, and journals, to construct a literary image of 
Attica to better understand the revolutionary, galvanizing effect 
the riots had on contempory society. 

I believe this is currently just as relevant as it was 50 years 
ago. Though we might not hear as much about current-day prison 
riots, the conditions for incarcerated peoples are still just 
as terrible as, if not worse than, they were at Attica: what little 
prison reform was passed in the 70's was later repealed in the 80's 
and 90's. Some of the sources I'm interested in consulting and using in my 
construction of Attica-Image are these:



\nocite{*}

\printbibliography


\end{document}
