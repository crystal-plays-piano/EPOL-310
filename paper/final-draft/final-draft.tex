
\documentclass[14pt, letterpaper]{report}
\usepackage[utf8]{inputenc}

% - default packages
\usepackage[backend=biber,style=apa]{biblatex}
\usepackage[doublespacing]{setspace}
\usepackage{indentfirst}
\usepackage{parskip}
\usepackage{amsmath}
\usepackage{changepage}
% -\usepackage{fancyhdr}
% \usepackage[dvipsnames]{xcolor}

% - set variables
% - \setlength{\parindent}{8ex}
% \definecolor{house-blue}{RGB}{0, 71, 187}



% - bibliography packages
% \usepackage[american]{babel}
%	\addto{\captionsenglish}{\renewcommand{\bibname}{Works that Inspired this Essay}}
% \usepackage{hyperref}
 \usepackage{csquotes}

% - import bib file
\addbibresource{../bibliography.bib}

% - commands

% - title

\title{ \vspace*{-72pt} Literary Construction of Attica in Post-Riot Cultural Products }
\author{Crystal Mandal}
\date{}

\begin{document}



\maketitle

\section*{Introduction}

	On the 5th of November, 2025, members of the California 
	public voted on Proposition 6. According to the California 
	General Voter's Guide, Proposition 6 ends slavery by 
	``replacing involuntary carceral servitude  with voluntary work 
	programs". The bill ran unopposed but still failed among the 
	public: that is, California voted against an abolishment of 
        slavery.
	
	The primary identifier in the wording of Proposition 
	6 is ``carceral" - of or relating to the nature of prisons. 
	What about the nature of prisons justifies a contemporary,
	protected installation of slavery?
	
	Sharon Luk's ``The Life of Paper" details a framework of 
	incarceration that establishes the use of threat of incarceration 
	as silencer of dissenting voices. The Governing Power constructs 
	the prison such that the mere threat of incarceration is a 
	policing force. There is a great deal of conversation constructing 
	the prison as an ideal in academic and social contexts with a 
	``top-down" or "subtractive" model (by starting with a general 
	concept and imposing restrictions and filters to increase the 
	resolution). In this literary exploration, I wish to construct a 
	framework of (contemporary, American) prisons with a ``bottom-up" 
	or "granular/additive" model (that is, by starting with a sample 
	set of ``grains" and modulating, interpolating between, and 
	resampling them to produce a model) by examining poems, letters, 
	and pieces of music to generate a cultural image of Prison.
	
	Now, the image of the American Prison System is massive - 
	and quite unfeasible to construct in a short exploration. From 
	the 9th of September, 1971, to the 13th of September, 1971, the 
	Attica State Prison Riot was publicised in such a meaningful way 
	that New York State Governor Rockefeller delayed police action away 
	from prime television hours to minimise viewing of the atrocities. 
	In following years in America, Attica remained a primary image of 
	the American Prison, and still remains culturally relevant, with 
	recent Television Show ``Orange is the New Black" Season 5 both 
	referencing directly Attica and paralleling the chronology of the 
	Attica Riots. The massive impact of the Attica Riots on contemporaneous 
	political and artistic movements (especially in the American Folk 
	Revival) as well as in contemporary cultural landscapes (with references 
	in ``Orange is the New Black" and, though a little older, still 
	relevant and beloved ``If I Ruled the World (Imagine That) by NAS) 
	cements Attica as a representative singular image of The American 
	Prison. 
	
	If Attica is representative of The American Prison, then construction 
	of an image of Attica is representative of the cultural image of 
	The American Prison. By analysing, relating, and resampling the 
	cultural response to Attica in 1970's America, we can begin to 
	construct an contemporary image of The Prison. In this exploration, 
	I will analyse the depiction of Attica in the Prison Letters of 
	Samuel Melville, the Music of Frederic Rzewski, and the published 
	Poems of Attica Inmates post-Riots, and use the underlying connecting 
	strands to fabricate a new, ``bottom-up" construction of Attica. 
	
	\section*{Preambulum - On Attica}
	
	Of vital importance to the construction of this image is, at first, 
	an understanding of the realities of the conditions and events at 
	Attica Correctional Facility. The Attica Correctional Facility is a 
	maximum security prison facility located in Attica, New York, about 
	40 miles east of Buffalo - and, more important to the majority of 
	Attica inmates, about 340 miles and 6 hours Northeast of New York 
	City. This distance cannot be travelled without a car; even today, 
	there is no public bus route from New York City to Attica, with only 
    privately operated shuttles offered. The standard trip, according to 
    prisonpulse dot com leaves Manhattan at 9:45 pm (21:45) the night before the 
    trip, and returns at 10:00 pm (22:00) the next day. The current price of 
    this trip is \$160USD: about 14 hours of minimum wage labor. 
    
    As detailled explicity in both the Official Report of the New York State Special Commision on 
    Attica (henceforth the Official Report) and Heather Ann Thompson's ``Blood in the 
    Water" - and referenced implicitly in Samuel Melvile's ``Letters from Attica" and 
    Celes Tisdale's poetry collections ``Betcha Ain't'' and ``When the Smoke Cleared" - 
    the makeup of the Attica prison population was heavily skewed towards one particular 
    socio-economic profile. Of Attica's nearly twenty-five hundred (2,500) inmates, forty 
    percent (40\%) were under the age of thirty, seventy-seven percent (77\%) were from 
    cities and predominantly urban areas, and sixty-three percent (63\%) were African 
    American or Puerto Rican. Eighty percent (80\%) of Attica's inmates circa September 1971 
    had not graduated high school. \autocite[580]{blood-in-water}
    These statistics point to a particular demographic of 
    inmate - a (relatively) un- or under- educated, nonwhite, urban, and younger man. 
    Thompson paints portraits of several represen	tative inmates, including: 
    
    \begin{itemize}
    
	    \item James and John Schleich - a pair of nineteen year old twins held in 
	    Attica for parole 
	    violations, with their initial convictions of ``unauthorized use of a motor 
	    vehicle" and ``cutting a hole in a lady's convertible top"
	    
	    \item Elliot ``L. D." Barker - a twenty-one year old inmate who was sent to 
	    Attica for driving without a license
	    
	    \item Angel Martinez - a seventeen year old Puerto Rican, who committed crimes to 
	    procure heroin while self-medicating for polio
	    
	    % \item 
    
    \end{itemize}
    
    The infamous article on Orientalism by Edward Said \autocite{said-orientalism} 
    introduces a core framework of analysis for non-physical geographies. Said's 
    ``imagined geography" is a criticism of the Orient: the Orient is not real, 
    and thus cannot have a location, but it's discoursed physicality imbues it with 
    a geography that cannot be outlined on any world map but in conversation is 
    nonetheless ``real". As Said explains, there is no symmetric field - an ``Occidentalism" 
    to study the ``Occident" - because the imagined geography of the Orient is only 
    defined by its quality as an Other, and to study and classify a non-Other (a ``norm") 
    would be silly. \autocite[163]{said-orientalism} We can use this framework to 
    discourse the physicality of the carceral image. Yes, in the case of Attica, the 
    location of the prison is far away from the cities
	
   \section*{Outline}
	
	\begin{enumerate}

		\item Preamble - On Attica (brief history + chronology)
		\begin{itemize}
			\item   racial makeup - 64\% black inmate population in D yard,\autocite[490]{attica-report}
				no black guards \autocite[147]{letters-from-attica}

			\item   inhumane living conditions - low temperatures, minimal activity, poor 
				hygiene offerings, insufficient food\autocite{blood-in-water}
			
			\item   govermnment censorship + obfuscation\autocite[573]{blood-in-water}
		\end{itemize}
	
		\item On Letters
		\begin{itemize}
		
			\item Discussion of Luk's ``Life of Paper"
			\begin{itemize}
				
				\item Letter as Voice
				\begin{itemize}
					
					\item ``one's habits and abilities are judged 
						by his letters"\autocite[2]{life-of-paper}

					\item ``This is what I think: people don't write to
						a prisoner either out of indifference or because
						of a lack of imagination"\autocite[6]{life-of-paper}

					\item ``it don't come out near what i want.
						in four tries on a letter to kenny i still havn't sent 
						anything. \ldots he's just not a person with whom one 
						has verbal communication"\autocite[87]{letters-from-attica}

					\item ``I'm sorry if some of this is illegible. I wrote if off the top 
						and I don't really have much to say, evidently."\autocite[144]{letters-from-attica}			
					
				\end{itemize}

				\item Letter in Prisons
				\begin{itemize}
		
					\item ``My dear wife, As the Japanese censor is away again, I write this in 
						English"\autocite[121]{life-of-paper}
					
				\end{itemize}
				
				\item Systematic Censorship of Writing
				\begin{itemize}

					\item ``affect as mode of historical intervention", 
					``prohibitions on formal self-representation and by 
					dominant reproductions of selfhood as an autonomous 
					rational subject"\autocite[121]{life-of-paper}

				
				\end{itemize}
				
			\end{itemize}

			\item   Now that letters have been introduced as substantial, subtext-heavy, 
			        pieces of writing, introduce Jackson + Cleaver
				\begin{itemize}
				
					\item On Becoming
					\begin{itemize}
						\item ``Of course I'd always known that I was black, but I'd
							never really stopped to take stock of what I was involved 
							in."\autocite[3]{soul-on-ice}
						\item ``I defied the law and they put my in prison. So why not put
							those dirty mothers in prison too?"\autocite[4]{soul-on-ice}
						\item ``All I could recall was an eternity of pacing back and forth in
							the cell, preaching to the unhearing walls"\autocite[11]{soul-on-ice}
						\item ``That is why I started to write. To save myself."\autocite[15]{soul-on-ice}
					\end{itemize}
					
					\item Similar elaborations on Cleaver's ``Soul Food" and, especially, ``A Day in Folsom Prison"

					\item   Jan 12 1967 - ``Your Letter was well received; it left me feeling better 
						than I have felt for years. I have never felt as close to any human as I 
						do to you now."\autocite[99]{soledad-brother}

					\item   Jan 23 1967 - ``I tried to write several times these last couple of weeks 
						but my letters all came back with a note attached explainint what I can and 
						cannot say."\autocite[101]{soledad-brother}

					\item   Oct 17 1967 - ``I suffer a constant bombardment of nonsense from all 
						sides."\autocite[139]{soledad-brother}
						\begin{itemize}
							\item interesting parallel with ``[in the] ravings of lost hysterical
							men i can act with clarity and meaning"\autocite[110]{letters-from-attica}
							(this text is also represented in coming together, by frederic rzewski)\autocite{coming-together}
						\end{itemize} 

				\end{itemize}
		
		\end{itemize}			

		\item   Poems
			\begin{itemize}
				
				\item The poems collected in \textit{Betcha Ain't} and \textit{When the Smoke Cleared} by Celes Tisdale
				
				\item text of \textit{Coming Together} and \textit{Attica}

				\item text of \textit{If I Ruled the World}				
		
			\end{itemize}

		\item   Music
			\begin{itemize}
				\item Analysis of \textit{Coming Together} and \textit{Attica}\autocite{prisoners-voices}
				\begin{itemize}
					\item   8x8 phrase  construction - small section length to depict claustrophobia of cells
					\item   rigid phrasing - rigid but asymmetric phrase length rules to depict rigid but 
						arbitrary policing and ruling by guards
					\item   repeating source material - ``[in the] ravings of lost hysterical men"
				\end{itemize}
				
				\item Personal analysis of Nas' \textit{If I Ruled the World}

			\end{itemize}

		\item   Granular Synthesis
			\begin{itemize}
				\item physical + mental brutality, 
				\item racialisation, 
				\item censorship, 
				\item geographical disconnection
			\end{itemize}
	
	\end{enumerate}

\clearpage

\nocite{*}

\printbibliography

\end{document} 
