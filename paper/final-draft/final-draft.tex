
\documentclass[14pt, letterpaper]{report}
\usepackage[utf8]{inputenc}

% - default packages
\usepackage[backend=biber,style=apa]{biblatex}
\usepackage[doublespacing]{setspace}
\usepackage{indentfirst}
\usepackage{parskip}
\usepackage{amsmath}
\usepackage{changepage}
% -\usepackage{fancyhdr}
% \usepackage[dvipsnames]{xcolor}

% - set variables
% - \setlength{\parindent}{8ex}
% \definecolor{house-blue}{RGB}{0, 71, 187}

\usepackage{titling}
\newcommand{\subtitle}[1]{%
  \posttitle{%
    \par\end{center}
    \begin{center}\large#1\end{center}
    \vskip0.5em}%
}




% - bibliography packages
% \usepackage[american]{babel}
%	\addto{\captionsenglish}{\renewcommand{\bibname}{Works that Inspired this Essay}}
% \usepackage{hyperref}
\usepackage{csquotes}

% - import bib file
\addbibresource{../bibliography.bib}

% - commands

% - title

\title{ \vspace*{-72pt} Attica Suite}
\subtitle{Construction of Attica-image in Post-Riot Cultural Products}
\author{Crystal Mandal}
\date{}

\begin{document}



\maketitle

	\section*{Introduction}

	On the 5th of November, 2025, members of the California 
	public voted on Proposition 6. According to the California 
	General Voter's Guide, Proposition 6 ends slavery by 
	``replacing involuntary carceral servitude  with voluntary work 
	programs". The bill ran unopposed but still failed among the 
	public: that is, California voted against an abolishment of 
        slavery.
	
	The primary identifier in the wording of Proposition 
	6 is ``carceral" - of or relating to the nature of prisons. 
	What about the nature of prisons justifies a contemporary,
	protected installation of slavery?
	
	Sharon Luk's ``The Life of Paper" details a framework of 
	incarceration that establishes the use of threat of incarceration 
	as silencer of dissenting voices. The Governing Power constructs 
	the prison such that the mere threat of incarceration is a 
	policing force. There is a great deal of conversation constructing 
	the prison as an ideal in academic and social contexts with a 
	``top-down" or "subtractive" model (by starting with a general 
	concept and imposing restrictions and filters to increase the 
	resolution). In this literary exploration, I wish to construct a 
	framework of (contemporary, American) prisons with a ``bottom-up" 
	or "granular/additive" model (that is, by starting with a sample 
	set of ``grains" and modulating, interpolating between, and 
	resampling them to produce a model) by examining poems, letters, 
	and pieces of music to generate a cultural image of Prison.
	
	Now, the image of the American Prison System is massive - 
	and quite unfeasible to construct in a short exploration. From 
	the 9th of September, 1971, to the 13th of September, 1971, the 
	Attica State Prison Riot was publicised in such a meaningful way 
	that New York State Governor Rockefeller delayed police action away 
	from prime television hours to minimise viewing of the atrocities. 
	In following years in America, Attica remained a primary image of 
	the American Prison, and still remains culturally relevant, with 
	recent Television Show ``Orange is the New Black" Season 5 both 
	referencing directly Attica and paralleling the chronology of the 
	Attica Riots. The massive impact of the Attica Riots on contemporaneous 
	political and artistic movements (especially in the American Folk 
	Revival) as well as in contemporary cultural landscapes (with references 
	in ``Orange is the New Black" and, though a little older, still 
	relevant and beloved ``If I Ruled the World (Imagine That) by NAS) 
	cements Attica as a representative singular image of The American 
	Prison. 
	
	If Attica is representative of The American Prison, then construction 
	of an image of Attica is representative of the cultural image of 
	The American Prison. By analysing, relating, and resampling the 
	cultural response to Attica in 1970's America, we can begin to 
	construct an contemporary image of The Prison. In this exploration, 
	I will analyse the depiction of Attica in the Prison Letters of 
	Samuel Melville, the Music of Frederic Rzewski, and the published 
	Poems of Attica Inmates post-Riots, and use the underlying connecting 
	strands to fabricate a new, ``bottom-up" construction of Attica. 
	
	\section*{Preambulum - }
	
	Of vital importance to the construction of this image is, at first, 
	an understanding of the realities of the conditions and events at 
	Attica Correctional Facility. The Attica Correctional Facility is a 
	maximum security prison facility located in Attica, New York, about 
	40 miles east of Buffalo - and, more important to the majority of 
	Attica inmates, about 340 miles and 6 hours Northeast of New York 
	City. This distance cannot be travelled without a car; even today, 
	there is no public bus route from New York City to Attica, with only 
    privately operated shuttles offered. The standard trip, according to 
    prisonpulse dot com leaves Manhattan at 9:45 pm (21:45) the night before the 
    trip, and returns at 10:00 pm (22:00) the next day. The current price of 
    this trip is \$160USD: about 14 hours of minimum wage labor. 
    
    As detailled explicity in both the Official Report of the New York State Special Commision on 
    Attica (henceforth the Official Report) and Heather Ann Thompson's ``Blood in the 
    Water" - and referenced implicitly in Samuel Melvile's ``Letters from Attica" and 
    Celes Tisdale's poetry collections ``Betcha Ain't'' and ``When the Smoke Cleared" - 
    the makeup of the Attica prison population was heavily skewed towards one particular 
    socio-economic profile. Of Attica's nearly twenty-five hundred (2,500) inmates, forty 
    percent (40\%) were under the age of thirty, seventy-seven percent (77\%) were from 
    cities and predominantly urban areas, and sixty-three percent (63\%) were African 
    American or Puerto Rican. Eighty percent (80\%) of Attica's inmates circa September 1971 
    had not graduated high school. \autocite[580]{blood-in-water}
    These statistics point to a particular demographic of 
    inmate - a (relatively) un- or under- educated, nonwhite, urban, and younger man. 
    Thompson paints portraits of several represen	tative inmates, including: 
    
    \begin{itemize}
    
	    \item James and John Schleich - a pair of nineteen year old twins held in 
	    Attica for parole 
	    violations, with their initial convictions of ``unauthorized use of a motor 
	    vehicle" and ``cutting a hole in a lady's convertible top",
	    
	    \item Elliot ``L. D." Barker - a twenty-one year old inmate who was sent to 
	    Attica for driving without a license,
	    
	    \item Angel Martinez - a seventeen year old Puerto Rican, who was incarcerated 
	    for crimes relating to his self-medication for polio with heroin use.
	    
	    % \item 
    
    \end{itemize}
    
    Of note is the non-violent nature of these crimes, the young age of the 
    inmates, and linguistic barriers; Puerto Rican Angel Martinez spoke only 
    Spanish, which made it impossible to communicate with an exclusively 
    English-speaking prison guard. \autocite[7]{blood-in-water}
    
    \section*{\textit{Soul on Ice} and \textit{Soledad Brother}}    
    
	A few years prior to the ticking time-bomb that is Attica in 1971, two 
	important incarcerated leaders of the Black Panther Party were writing 
	in California. It is essential to understanding the narrative of prison 
	abolition and carceral justice that one is familiar with the writings of 
	George Jackson and Eldridge Cleaver. Both Jackson and Cleaver were 
	incarcerated at the Correctional Training Facility near Soledad, California.
	Though they were contemporaries, they were not well acquainted. 
	Due to the nondescript naming and the location, this facility is more 
	commonly known as - both in this essay and in related writings - Soledad 
	State Prison. Cleaver was later also incarcerated at Folsom State Prison, 
	which is where much of his writing in \textit{Soul on Ice} comes from. 
	Similar to Attica, Soledad and Folsom were both heavily racialised in 
	their inmate demographics. Both primarily hold Black and Latine inmates.

	Written as an exercise, one of the articles collected in Cleaver's 
	\textit{Soul on Ice} is the excruciatingly detailled ``A Day in Folsom
	Prison", where he summarises the events of an average day in his life. 
	As he narrates, his day begins with a disciplined self-awakening at 
	5:30 am before 
	the officially scheduled awakening at 7:00 am. He 
	reorganises and cleans his cell, exercises for a little while, 
	takes a ``jailbird bath" in the sink in his cell, and listens to 
	the news on the radio - all before he is forcibly moved to the mess hall 
	for breakfast at 7:30 am. \autocite[64]{soul-on-ice} While it is unnecessary 
	to fully recount his day, it is important to note that his schedule is 
	heavily regimented and dependent on the will of the prison officers. 
	Another important statistic is his solitary time: in the course of a 
	usual day in Folsom Prison, Cleaver ``spend[s] approximately seventeen 
	hours a day in [his] cell". On average, according to the Pew Research 
	Center, a person living alone spends about ten (10) hours a day in 
	solitude. \autocite{pew-alone} Thus, Cleaver spends, on average, almost 
	double the amount of time alone, whilst still admitting that he makes 
	efforts to engage with others. The life of an inmate is strictly regulated and 
	solitary. Cleaver's only refuge is writing, particularly letters. 
	
	Indeed, letters 	tend to act as the primary contact an inmate has with the 
	outside world. George Jackson's foundational collection \textit{Soledad Brother} 
	is much less formal than Cleaver's \textit{Soul on Ice}. Where Cleaver contrasts 
	his (carefully chosen) letters with expository writing, essays, and manifestos, 
	\textit{Soledad Brother} is simply a collection of Jackson's prison letters. 
	In fact, the very first of these letters is the only one acknowledging such a 
	collection. The others are simply the message atoms of his incarcerated reality.
	
	\section*{\textit{Letters from Attica}}
	
	Both \textit{Soul on Ice} and \textit{Soledad Brother} are written in the 
	late 1960's - about five years prior to the events at Attica. Though the 
	content of these collections are relevant, the physical and chronological 
	distances place Cleaver and Jackson only adjacent to Attica - not far away,
	but not explicitly related. Of vital importance to this construction of 
	Attica-image is the vibrant and extremely controveersial collection 
	``Letters from Attica", written by Samuel Melville. 
	
	Samuel Melville is an odd figure in the narrative of Attica. He is described 
	as a terrorist in contemporaneous news articles and as the ``mad bomber [and]
	member of 5 company" in the Official Report. In his son's words, he is a 
	modern-day John Brown\autocite{atticas-ashes}, and in his own he is a 
	reformed honky.\autocite[53]{letters-from-attica} \footnotemark His famed 
	collection \textit{Letters from Attica} consists primarily of his letters 
	sent between 1969 and 1971, with some of his manifestos and articles peppered 
	chronologically between certain letters. The existence of this text is itself 
	an argument about the racial tensions and the flattening of narrative at 
	Attica. Due to Melville's active role in organising prior to and during 
	the riots and his passive role in being white, his actions and influences 
	were, at the time of the Official Report minimized and unfocused - it's 
	difficult to paint a conflict as a microcosm for race war when multiple 
	racialised factions (some of the same race) are fighting against each other 
	and a primary faction leader was white. Melville's letters will be discussed 
	at length later.
    
	\footnotetext{This is the first time that a racially charged term - a slur, 
	if you will - is used in this essay. Use of these charged terms is not done 
	for any frivolous purposes, but because the author believes that in racial 
	discourses these terms ``pop up", and it is ideologically dampening and 
	censoring to remove or obfuscate the usage of these terms in the related 
	literature. Further segments in this essay will again use these words. 
	As this is potentially upsetting, I would advise to please take note and be 
	aware.}   
	
    \section*{Geographies}
    
    The infamous article on Orientalism by Edward Said \autocite{said-orientalism} 
    introduces a core framework of analysis for non-physical geographies. Said's 
    ``imagined geography" is a criticism of the Orient: the Orient is not real, 
    and thus cannot have a location, but it's discoursed physicality imbues it with 
    a geography that cannot be outlined on any world map but in conversation is 
    nonetheless ``real". As Said explains, there is no symmetric field - an ``Occidentalism" 
    to study the ``Occident" - because the imagined geography of the Orient is only 
    defined by its quality as an Other, and to study and classify a non-Other (a ``norm") 
    would be meaningless. \autocite[163]{said-orientalism} We can use this framework to 
    discourse the physicality of the carceral image. Yes, in the case of Attica, the 
    location of the prison is far away from the cities, and this does contribute to 
    the image of ``the prison" as distant, but this is further articulated by the 
    use of letter as communication. As detailled by Sharon Luk in \textit{The Life 
    of Paper} and the USPS website, physical mail travels long distances over long 
    times to get to the recipient. The additional time and financial burden of 
    constructing and sending a message by the post imposes additional barriers to 
    communication. This restricted bandwidth facilitates the Othering of the sender; 
    in an era of instant and widely available, mobile calling, texting, and video 
    communication technologies, it is fundamentally easy to reach someone. This 
    unlimited bandwidth has the opposite effect of superficially familiarising all 
    perspectives. 
    
    As recounted in ``A Day in Folsom Prison", mail is to be posted at 8am - only 
    thirty minutes after the scheduled wake-up time.\autocite[64]{soul-on-ice}
     In certain Computer-Science 
    fields, an ``off-by-one" error occurs when two processes fail to properly 
    interact due to incongruent indexing systems. Here, if an inmate begins their
    day with breakfast, they have missed their window for mailing, and thus the 
    calendar of writing the letter and sending the letter are now displaced. 
    
    Additionally, there is much labor necessary for the production of the 
    letter. Other than the arduous task of putting intentions to thoughts 
    to words, one must take extreme care to evade censorship. In Luk's 
    \textit{The Life of Paper}, Luk analyses the structure of writing 
    from within Japanese internement in World War II America. While 
    it's beyond the scope of this exploration to truly discourse Luk's 
    analysis, it's helpful to at least be familiar. Luk defines censorship
    rather 
    eloquently as ``prohibitions on formal self-representation and
    dominant reproductions of selfhood as an autonomous rational 
    subject."\autocite[121]{life-of-paper} That is, censorship 
    not only restricts language use, but, because, as she describes 
    ``one's habits and abilities are judged by his letters", the 
    affect of one's writing informs how a subject constructs 
    itself.\autocite[2]{life-of-paper} As an example, thus begins 
    a letter excerpted in \textit{The Life of Paper}:
    
    \begin{quote}   
    		``My dear wife, As the Japanese censor is away again, I write this in 
    		English"\autocite[121]{life-of-paper}
    \end{quote}
	
	\section*{Quodlibet Fantasia - Excerpts Analysed}
	Now that a groundwork for discourse has been laid, the following sections 
	will deconstruct several very important excerpts.
	
	\subsection*{\textit{Coming Together}}
	\begin{quote} i think the combination of age and a greater coming 
	together is responsible for the speed of the passing time. it’s 
	six months now and i can tell you truthfully few periods in my life 
	have passed so quickly. i am in excellent physical and emotional health. 
	there are doubtless subtle surprises ahead but i feel secure and ready. 
	As lovers will contrast their emotions in times of crisis, so am i dealing 
	with my environment. in the indifferent brutality, incessant noise, the 
	experimental chemistry of food, the ravings of lost hysterical men, i 
	can act with clarity and meaning. i am deliberate - sometimes even 
	calculating - seldom employing histrionics except as a test of the 
	reactions of others. i read much, exercise, talk to guards and inmates, 
	feeling for the inevitable direction of my life.\autocite{coming-together}
	
	\end{quote}
	
	This passage by Samuel Melville is remarkably vivid and dense. The rhythmic
	pulse is deliberate - calculating - in its disorienting nature. Samuel Melville 
	was known for utilising unique sentence structures and capitalisation schema 
	for different tones and voices. Here, the lack of capitalisation serves to 
	save ink and effort, but also act as a subtle reminder that ``the indifferent 
	brutality, incessant noise, experimental chemistry of food, [and] the 
	ravings of lost, hysterical men" all take their toll on the human spirit. In 
	setting this to music, Rzewski chose this text for it's haunting nature. As 
	described in Metzer's \textit{Prisoner's Voices}, Rzewski's setting is a 
	minimalist piece with asymmetric phrase length and text setting and a 
	very unusual and rigid structure. Though it's not important to 
	get into the dense musical analysis, it is quite helpful in understanding the 
	suffocating nature of this work. The constructive atoms that we can use from 
	\textit{Coming Together} primarily describe the physical and mental toll the 
	repetitive prison environment has on the inmate. It is most important to 
	recognise ```the incessany noise" and ``indifferent brutality", as well as the 
	expedited passing of time as the atomic elements of \textit{Coming Together}.
	
	\subsection*{\textit{On Becoming}}
	
	\begin{quote}
	``Of course I'd always known that I was black, but I'd never really 
	stopped to take stock of what I was involved in."\autocite[3]{soul-on-ice}
	\end{quote}		
	
	\begin{quote}
	``All I could recall was an eternity of pacing back and forth in 
	the cell, preaching to the unhearing walls"\autocite[11]{soul-on-ice}
	\end{quote}
	
	\begin{quote}
	``That is why I started to write. To save myself."\autocite[15]{soul-on-ice}
	\end{quote}
	
	Cleaver's \textit{On Becoming} from \textit{Soul on Ice} demonstrates a
	particular racialisation of carceral oppression. The letter begins with 
	a striking thought: until Cleaver's experience with incarceration, being 
	black wasn't as relevant to his life. Cleaver's depictions in \textit{A 
	Day in Folsom Prison} paint his prison life as regimented and soul-destroying, 
	yes, but his conversations in \textit{On Becoming} shape them as distinctly 
	Black experience. Also illuminating is his reassurance of Luk's claim that 
	``ones habits and abilities are judged by his letters".\autocite[2]{life-of-paper}
	When Cleaver decides that it is time to do something in Hell, he chooses to 
	write.
	
	\subsection*{\textit{Soledad Brother}}
	
	\begin{quote}
	Jan 12 1967 - ``Your Letter was well received; it left me feeling better 
	than I have felt for years. I have never felt as close to any human as I 
	do to you now."\autocite[99]{soledad-brother}
	\end{quote}
	
	\begin{quote}
	Jan 23 1967 - ``I tried to write several times these last couple of weeks 
	but my letters all came back with a note attached explaining what I can and 
	cannot say."\autocite[101]{soledad-brother}
	\end{quote}

	\begin{quote}
	Oct 17 1967 - ``I suffer a constant bombardment of nonsense from all 
	sides."\autocite[139]{soledad-brother}
	\end{quote}
	
	Jackson's letters perfectly and concisely demonstrate the imagined distances 
	between inmates and the outside world. To Jackson, the humble letter is the 
	closest contact he has to anyone else. How far must the Prison be that the 
	distant letter - as discussed earlier in this exploration - is reimagined as 
	``close"? Further, his writing is displaced even more from his own voice by 
	the restrictive censorship. It's enlightening and frightening to note the 
	similarity in language between Melville's ``incessant noise [and] ... ravings 
	of lost hysterical men" amd Jacksons ``constant bombardment of nonsense". 
	
	
	\subsection*{\textit{When the Smoke Cleared}}
	\begin{quote}
	\textbf{1st Page}\\
	They say our isolation is justifiable\\
	So, when I'm released\\
	I'll find a house or hut to live in\\
	In a lonely countryside\\
	\hspace*{8ex}With Atticka on my mind.\\
	- Daniel Brown\autocite[36]{attica-poems}
	\end{quote}
	
	\begin{quote}
	\textbf{Just Another Page}\\
	(September 13, 2972)\\
	A year later\\
	And it's just anoother page\\
	And the only thing they do right is wrong\\\
	And Attica is a maggot-minded black blood sucker\\
	And the only thing they do right is wrong\\
	And another page of history is written in black blood\\
	And old black mamas pay taxes to buy guns that killed their sons\\
	And the consequence of being free\ldots is death\\
	And your sympathy and tears always come too late\\
	And the only thing they do right is wrong\\
	\hspace*{16ex}And it's just another page.\\
	- John Lee Norris\autocite[52]{attica-poems}
	\end{quote}
	
	Finally, at the end of our journey, we can appreciate and dissect 
	the pearls of meaning from the poets (inmates) at Attica. These two 
	poems were written in Attica in 1972, just a year after the riots. 
	Much of the hostility towards the Officers and the revolutionary 
	sentiment is still prevalent, and the imagery is very vivid and 
	direcct. There is nothing Romantic about the pastoral dream in 
	\textbf{1st Page}. Brown's careful choice of the word ``hut" is 
	also quite important. Even when the inmate is released from prison, 
	the experience is so haunting that the inmate will continue to relive 
	it for years. The act of finding a house or a hut is written passively 
	to disconnect the outside, ``free" life from the eternal shadow of 
	incarceration. Brown's subject need not find a hut - the inmate may 
	be eligible for and strive toward any type of housing they wish - but 
	is doomed to find a cell only slightly better than the one at Attica. 
	
	For Norris, the Attica Riots were simply a passing event. Norris 
	corroborates earlier Jackson and Cleaver's notions that the Prison 
	runs off of Black exploitation - ``black blood". Further, Norris goes 
	on to critique the system that perpetuates Black grief and violence.
	The taxes that Black people pay simply supports the corrupt, anti-Black 
	police (and thus, incarceration) systems in America. For the Black 
	inmate - before or after incarceration - being free is Death. 
	
	\section*{Coda}
	
	All of the above discourse can be made more direct and concise. To 
	better understand and navigate a contemporary Prison-image, it is 
	sufficient to understand Attica-image ca. 1972. As argued by Luk 
	and Metzer, the song, poem, and letter are information-dense messages, 
	and are thus perfect cultural products to encode information about 
	Attica as subtext. The letters of Melville, Cleaver, and Jackson all 
	make very clear that Prison enforces harsh living conditions on the 
	inmates, especially with regard to self-regulation and determination. 
	Cleaver and Jackson further motion that Prison treats the inmate as 
	Black, and that the Black inmate is stripped of all power as a man. 
	Finally, the poems from Attica inmates highlight the residual impact 
	of their experiences at Attica and the Riots. 
	
	Though much is left to discuss regards Attica and Attica-image, 
	the above work constructs and clarifies a very basic image from 
	fundamental creative atoms as analysed from letters, essays, songs, 
	and poems. The conversation is here best ended with a story and a quote.
	
	In 1971, after the Riots, as Black Activist Richard X. Clark was 
	being escorted out of Attica and past Attica village limits, he 
	was asked how it felt to put Attica behind him. He responded 
	
	\begin{quote}
	
		``Attica is in front of me." - 	Richard X. Clark
	
	\end{quote}
	
%    \section*{Outline}
%	
%	\begin{enumerate}
%
%%		\item Preamble - On Attica (brief history + chronology)
%%		\begin{itemize}
%%			\item   racial makeup - 64\% black inmate population in D yard,\autocite[490]{attica-report}
%%				no black guards \autocite[147]{letters-from-attica}
%%
%%			\item   inhumane living conditions - low temperatures, minimal activity, poor 
%%				hygiene offerings, insufficient food\autocite{blood-in-water}
%%			
%%			\item   govermnment censorship + obfuscation\autocite[573]{blood-in-water}
%%		\end{itemize}
%	
%		\item On Letters
%		\begin{itemize}
%		
%			\item Discussion of Luk's ``Life of Paper"
%			\begin{itemize}
%				
%				\item Letter as Voice
%				\begin{itemize}
%					
%%					\item ``one's habits and abilities are judged 
%%						by his letters"\autocite[2]{life-of-paper}
%%
%%					\item ``This is what I think: people don't write to
%%						a prisoner either out of indifference or because
%%						of a lack of imagination"\autocite[6]{life-of-paper}
%
%					\item ``it don't come out near what i want.
%						in four tries on a letter to kenny i still havn't sent 
%						anything. \ldots he's just not a person with whom one 
%						has verbal communication"\autocite[87]{letters-from-attica}
%
%					\item ``I'm sorry if some of this is illegible. I wrote if off the top 
%						and I don't really have much to say, evidently."\autocite[144]{letters-from-attica}			
%					
%				\end{itemize}
%
%				\item Letter in Prisons
%%				\begin{itemize}
%%		
%%					\item ``My dear wife, As the Japanese censor is away again, I write this in 
%%						English"\autocite[121]{life-of-paper}
%%					
%%				\end{itemize}
%%				
%%				\item Systematic Censorship of Writing
%%				\begin{itemize}
%%
%%					\item ``affect as mode of historical intervention", 
%%					``prohibitions on formal self-representation and by 
%%					dominant reproductions of selfhood as an autonomous 
%%					rational subject"\autocite[121]{life-of-paper}
%%
%%				
%%				\end{itemize}
%				
%			\end{itemize}
%
%			\item   Now that letters have been introduced as substantial, subtext-heavy, 
%			        pieces of writing, introduce Jackson + Cleaver
%				\begin{itemize}
%				
%%					\item On Becoming
%%					\begin{itemize}
%%						\item ``Of course I'd always known that I was black, but I'd
%%							never really stopped to take stock of what I was involved 
%%							in."\autocite[3]{soul-on-ice}
%%						\item ``I defied the law and they put my in prison. So why not put
%%							those dirty mothers in prison too?"\autocite[4]{soul-on-ice}
%%						\item ``All I could recall was an eternity of pacing back and forth in
%%							the cell, preaching to the unhearing walls"\autocite[11]{soul-on-ice}
%%						\item ``That is why I started to write. To save myself."\autocite[15]{soul-on-ice}
%%					\end{itemize}
%					
%					\item Similar elaborations on Cleaver's ``Soul Food" and, especially, ``A Day in Folsom Prison"
%
%					\item   Jan 12 1967 - ``Your Letter was well received; it left me feeling better 
%						than I have felt for years. I have never felt as close to any human as I 
%						do to you now."\autocite[99]{soledad-brother}
%
%					\item   Jan 23 1967 - ``I tried to write several times these last couple of weeks 
%						but my letters all came back with a note attached explainint what I can and 
%						cannot say."\autocite[101]{soledad-brother}
%
%					\item   Oct 17 1967 - ``I suffer a constant bombardment of nonsense from all 
%						sides."\autocite[139]{soledad-brother}
%						\begin{itemize}
%							\item interesting parallel with ``[in the] ravings of lost hysterical
%							men i can act with clarity and meaning"\autocite[110]{letters-from-attica}
%							(this text is also represented in coming together, by frederic rzewski)\autocite{coming-together}
%						\end{itemize} 
%
%				\end{itemize}
%		
%		\end{itemize}			
%
%		\item   Poems
%			\begin{itemize}
%				
%				\item The poems collected in \textit{Betcha Ain't} and \textit{When the Smoke Cleared} by Celes Tisdale
%				
%%				\item text of \textit{Coming Together} and \textit{Attica}
%
%				\item text of \textit{If I Ruled the World}				
%		
%			\end{itemize}
%
%%		\item   Music
%%			\begin{itemize}
%%				\item Analysis of \textit{Coming Together} and \textit{Attica}\autocite{prisoners-voices}
%%				\begin{itemize}
%%					\item   8x8 phrase  construction - small section length to depict claustrophobia of cells
%%					\item   rigid phrasing - rigid but asymmetric phrase length rules to depict rigid but 
%%						arbitrary policing and ruling by guards
%%					\item   repeating source material - ``[in the] ravings of lost hysterical men"
%%				\end{itemize}
%%				
%%				\item Personal analysis of Nas' \textit{If I Ruled the World}
%%
%%			\end{itemize}
%
%		\item   Granular Synthesis
%			\begin{itemize}
%				\item physical + mental brutality, 
%				\item racialisation, 
%				\item censorship, 
%				\item geographical disconnection
%			\end{itemize}
%	
%	\end{enumerate}

\clearpage

\nocite{*}

\printbibliography

\end{document} 
