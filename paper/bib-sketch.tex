

\fullcite{attica-report}

\begin{adjustwidth}{2.5ex}{0pt}

  The New York Special Commission Report on Attica (hereafter and in 
  further writing simply the "Attica Report") is a very special and 
  controversial document. It was written in the year following the 
  Riots and published one year later in September of 1972. The Report 
  seems a reasonable and complete documentation of the events of the 
  Riots, with special consideration to the events as they chronologically 
  occurred and the justification of force of the Prison Guards. In 
  essence, however, the function of this report is not to inform the 
  public but rather to propagandize. Though the report details 
  the conditions at Attica, the aim of the report is to justify these 
  conditions, not to critique. As such, much of the report is dedicated 
  to \textit{what} happened and \textit{how} it happened and 
  \textit{when} it happened, but little about \textit{\textbf{why}} 
  it happened. This report is quite helpful to understand the scale 
  and physical and legal impact of the Riots, as well as being one 
  piece that cements prison as an othered, alien state. 

\end{adjustwidth}


\clearpage


\fullcite{blood-in-water}

\begin{adjustwidth}{2.5ex}{0pt}

  Heather Ann Thompson's Blood in the Water is simply what the 
  Attica Report could have been were it written to inform, not to 
  propagandize - and also if it were written 45 years later. Blood 
  in the Water is a comprehensive report on the when, where, how, what, 
  and, most importantly, \textit{why} of the events that occured at 
  Attica during the riots. The Attica Report starts with an encounter 
  between inmate Dewer and Officers Maronie and Curtiss. Though 
  the encounter is equally described in both the official Report and 
  in Blood in the Water, Thompson's recording focuses primarily on 
  the voice of the prisoners and the dissenting officers. By the 
  publication date of Blood in the Water, it is well understood 
  that the brutality of incarceration is obscured in the public 
  eye, and that the official reports on Attica (as assured by 
  Joshua Melville in Attica's Ashes) was in many ways 
  intentionally inconclusive - particularly in the racial makeup of the 
  different factions in the Attica riots and the militance of several 
  "celebrity" figures. I plan to use Blood in the Water as a way to 
  "check and balance" the information and the records in the official 
  Attica Report, as well as to reinforce the idea of oppression of the 
  prison image. 

\end{adjustwidth}


\clearpage


\fullcite{coming-together}

\begin{adjustwidth}{2.5ex}{0pt}

  Frederic Rzewski was a very unique musical voice. Much of his music 
  (as emphasized in David Metzer's "Prisoner's Voices") is politically 
  charged, contemporary, and specifically related to anti-war and 
  prison abolition movements.

\end{adjustwidth}
