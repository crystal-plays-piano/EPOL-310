
\documentclass[14pt, letterpaper]{report}
\usepackage[utf8]{inputenc}

% - default packages
\usepackage[backend=biber, style=apa]{biblatex}
\usepackage[doublespacing]{setspace}
\usepackage{indentfirst}
\usepackage{parskip}
\usepackage{amsmath}
\usepackage{changepage}
% \usepackage[dvipsnames]{xcolor}

% - set variables
% - \setlength{\parindent}{8ex}
% \definecolor{house-blue}{RGB}{0, 71, 187}



% - bibliography packages
% \usepackage[american]{babel}
%	\addto{\captionsenglish}{\renewcommand{\bibname}{Works that Inspired this Essay}}
% \usepackage{hyperref}
 \usepackage{csquotes}

% - import bib file
\addbibresource{bibliography.bib}

% - commands

% - title

\title{ \vspace*{-72pt} Literary Construction of Attica in Post-Riot Cultural Products }
\author{Crystal Mandal}
\date{}

\begin{document}

\maketitle

\section*{Introduction}

	On the 5th of November, 2025, members of the California 
	public voted on Proposition 6. According to the California 
	General Voter's Guide, Proposition 6 ends slavery by 
	``replacing involuntary carceral servitude  with voluntary work 
	programs". The bill ran unopposed but still failed among the 
	public: that is, California voted against an abolishment of 
	slavery.
	
	The primary identifier in the wording of Proposition 
	6 is ``carceral" - of or relating to the nature of prisons. 
	What about the nature of prisons justifies a contemporary,
	protected installation of slavery?
	
	Sharon Luk's ``The Life of Paper" details a framework of 
	incarceration that establishes the use of threat of incarceration 
	as silencer of dissenting voices. The Governing Power constructs 
	the prison such that the mere threat of incarceration is a 
	policing force. There is a great deal of conversation constructing 
	the prison as an ideal in academic and social contexts with a 
	``top-down" or "subtractive" model (by starting with a general 
	concept and imposing restrictions and filters to increase the 
	resolution). In this literary exploration, I wish to construct a 
	framework of (contemporary, American) prisons with a ``bottom-up" 
	or "granular/additive" model (that is, by starting with a sample 
	set of ``grains" and modulating, interpolating between, and 
	resampling them to produce a model) by examining poems, letters, 
	and pieces of music to generate a cultural image of Prison.
	
	Now, the image of the American Prison System is massive - 
	and quite unfeasible to construct in a short exploration. From 
	the 9th of September, 1971, to the 13th of September, 1971, the 
	Attica State Prison Riot was publicised in such a meaningful way 
	that New York State Governor Rockefeller delayed police action away 
	from prime television hours to minimise viewing of the atrocities. 
	In following years in America, Attica remained a primary image of 
	the American Prison, and still remains culturally relevant, with 
	recent Television Show ``Orange is the New Black" Season 5 both 
	referencing directly Attica and paralleling the chronology of the 
	Attica Riots. The massive impact of the Attica Riots on contemporaneous 
	political and artistic movements (especially in the American Folk 
	Revival) as well as in contemporary cultural landscapes (with references 
	in ``Orange is the New Black" and, though a little older, still 
	relevant and beloved ``If I Ruled the World (Imagine That) by NAS) 
	cements Attica as a representative singular image of The American 
	Prison. 
	
	If Attica is representative of The American Prison, then construction 
	of an image of Attica is representative of the cultural image of 
	The American Prison. By analysing, relating, and resampling the 
	cultural response to Attica in 1970's America, we can begin to 
	construct an contemporary image of The Prison. In this exploration, 
	I will analyse the depiction of Attica in the Prison Letters of 
	Samuel Melville, the Music of Frederic Rzewski, and the published 
	Poems of Attica Inmates post-Riots, and use the underlying connecting 
	strands to fabricate a new, ``bottom-up" construction of Attica. 
	
	\section*{Outline}
	
	\begin{enumerate}
	
		\item Letters
		\begin{enumerate}
		
			\item Discussion of Luk's ``Life of Paper
			\begin{itemize}
				
				\item Letter as Voice
				
				\item Letter in Prisons
				
				\item Systematic Censorship of Writing
				\begin{itemize}

					\item ``affect as mode of historical intervention", 
					``prohibitions on formal self-representation and by 
					dominant reproductions of selfhood as an autonomous 
					rational subject"
				
				\end{itemize}
				
			\end{itemize}
		
		\end{enumerate}			
	
	\end{enumerate}

\clearpage

\nocite{*}

\printbibliography

\end{document} 