
\documentclass[14pt, letterpaper]{article}
\usepackage[utf8]{inputenc}

% - default packages
\usepackage[backend=biber, style=apa]{biblatex}
\usepackage[doublespacing]{setspace}
\usepackage{indentfirst}
\usepackage{parskip}
\usepackage{amsmath}
\usepackage{changepage}
% \usepackage[dvipsnames]{xcolor}

% - set variables
% - \setlength{\parindent}{8ex}
% \definecolor{house-blue}{RGB}{0, 71, 187}



% - bibliography packages
% \usepackage[american]{babel}
%	\addto{\captionsenglish}{\renewcommand{\bibname}{Works that Inspired this Essay}}
% \usepackage{hyperref}
 \usepackage{csquotes}

% - import bib file
\addbibresource{../bibliography.bib}

% - commands

% - title

\title{ \vspace*{-72pt} Literary Construction of Attica in Post-Riot Cultural Products }
\author{Crystal Mandal}
\date{}

\begin{document}

\maketitle

\fullcite{soul-on-ice}

\begin{adjustwidth}{2.5em}{0pt}

	Soul on Ice is a collection of Letters and Essays by Black Panther 
	party member Eldridge Cleaver. The text was written from Folsom 
	State Prison in 1965: 6 years before the Attica Riots. The chronology 
	of carceral abuse from 1965 to 1973 can be traced through the writings  
	in this book and through the writings of Sam Melville's later 
	Letters from Attica. I plan to use primarily the ``Letters from 
	Prison" section and the ``Prelude to Love" section. The Letters ``On 
	Becoming", ``Soul Food", and ``A Day in Folsom Prison" can be used 
	to better understand Cleaver's reform in prison, with ``A Day in Folsom 
	Prison" specifically an itinerary and reflection on the brutality 
	of life in Folsom. Important to note is the harsh nature of both 
	Cleaver's writing style and content. This book has been subject of 
	a Supreme Court case about book banning in educational institutions and 
	libraries due to its graphic content. I will not be censoring the parts 
	of this text I plan to use, though I will not necessarily be discussing 
	the parts about gendered violence that garnered much controversy when it 
	was first published.

\end{adjustwidth}

\clearpage

\fullcite{soledad-brother}

\begin{adjustwidth}{2.5ex}{0pt}

	George Jackson's Soledad Brother is a much more direct and unfiltered
	work in comparison to the equivalent Letters from Attica and Soul 
	on Ice by Sam Melville and Eldridge Cleaver respectively. Soledad 
	Brother features a shorter introduction and exclusively epistolary 
	writing: Jackson inserts no essays, manifestos, or articles into 
	this collection. The product is a concentrated look into prison 
	life at Soledad State Prison and into the minds of a growing Black 
	Power movement. Most interesting is the similarity in language of 
	letters dated in 1970 to the language used by Melville in his 1970 
	letters, though the two never met or spoke. I plan to use snippets 
	of Jackson's text as evidence for racial institutional bias and 
	violence in the prison system.

\end{adjustwidth}

\clearpage

\fullcite{life-of-paper}

\begin{adjustwidth}{2.5ex}{0pt}

	Sharon Luk's The Life of Paper is a remarkable - though painfully dense - 
	work narrating The Letter's role as voice and as communication line for 
	incarcerated peoples. Chapters Two (of Genealogy and Diaspora), Four 
	(of Censorship), Five (of Ephemera), and Six (of Profanity). The work's 
	complexity is just barely comprehensible, though it offers glimpses of 
	truly life-altering revelations. I plan to reconstruct my understanding of 
	her epistolary framework to apply to Jackson, Cleaver, Melville, and 
	Tisdale's Letters, Essays, and Poems to illustrate the prevalence of the 
	figure of the prison in their writing. By illustrating the figure of the 
	prison, I can then compare it with the depictions of prison in 
	contemporaneous pieces like Rzewski's Coming Together, as well as historic 
	pieces like Oscar Wilde's famous De Profundis. 

\end{adjustwidth}

\clearpage

\fullcite{letters-from-attica}

\begin{adjustwidth}{2.5em}{0pt}

	Letters from Attica is a collection of Letters, Essays, and 
	Newspaper articles written by political prisoner Samuel Melville 
	during his time served in various American prisons. The Collection 
	is preceded by a statement from Samuel's son, Joshua Melville, 
	who reflects on the difficulty of collecting and printing 
	these letters, as well as the Government censorship surrounding 
	the Attica Prison Riots. I plan to use some of Sam's letters as 
	creative ``grains" in constructing an image of the "indifferent 
	brutality" of pre-Attica-Riot living conditions for the incarcerated.
	Highlights include a particularly poetic and resonant letter from 
	16th May, 1970,  detailling Melville's experience at ``The Tombs" 
	(a nickname for Manhatton Detention Complex.), a report entitled 
	``An Anatomy of the Laundry", and snippets of a section Melville 
	published in the Attica Newsletter titled ``The Iced Pig". 

\end{adjustwidth}

\clearpage

\fullcite{prisoners-voices}

\begin{adjustwidth}{2ex}{0pt}

  This source is a unique perspective of a Musicologist. Here, Metzer 
  argues that the construction of Rzewski's music is itself discourse 
  on prison architecture, image, and abolition. Though Metzer indicates 
  that much of Rzewski's music is relevant to this conversation, he 
  focuses primarily on the works ``Coming Together" and ``Attica", which 
  are uniquely related in that they are both written about and in the 
  aftermath of the Attica State Prison Riots. He argues that the minimalist, 
  repeating musical structure of the pieces is representative of the mental 
  landscape of one in isolation, leading to a restless \textit{moto perpetuo}
  in ``Coming Together" and a still reflection in ``Attica". Metzer's work 
  is foundational in contextualising music as informationally dense and 
  a strong carrier of political messaging. 
  
\end{adjustwidth}

\clearpage

\fullcite{attica-report}

\begin{adjustwidth}{2.5ex}{0pt}

  The New York Special Commission Report on Attica (hereafter and in 
  further writing simply the ``Attica Report") is a very special and 
  controversial document. It was written in the year following the 
  Riots and published one year later in September of 1972. The Report 
  seems a reasonable and complete documentation of the events of the 
  Riots, with special consideration to the events as they chronologically 
  occurred and the justification of force of the Prison Guards. In 
  essence, however, the function of this report is not to inform the 
  public but rather to propagandize. Though the report details 
  the conditions at Attica, the aim of the report is to justify these 
  conditions, not to critique. As such, much of the report is dedicated 
  to \textit{what} happened and \textit{how} it happened and 
  \textit{when} it happened, but little about \textit{\textbf{why}} 
  it happened. This report is quite helpful to understand the scale 
  and physical and legal impact of the Riots, as well as being one 
  piece that cements prison as an othered, alien state. 

\end{adjustwidth}

\clearpage

\fullcite{oswald-attica}

\begin{adjustwidth}{2.5ex}{0pt}

  ``I was the man in charge at Attica" - thus begins the memoir 
  of Russel G. Oswald and a tome of evading blame and virtue 
  signalling. It is true that Oswald, both in his story and in 
  his actions, was and claimed to be a prison reformer. He spends 
  the first chapter of his book denying his responsibility in the 
  massacre and fallout of the riots by minimising his effect to 
  six \textit{decisions} and spends the rest of the book justifying 
  said decisions at the blame of the prisoners and a ``new breed" of 
  radical inmates. Oswald's memoir on Attica, because of his biases, 
  serves as a perfect example of the conspiracy and narrative 
  displacement and censorship that exists at the heart of the image of 
  Attica in contemporaneous cultural discourse. I'm most interested in 
  dissecting his references to Abraham Lincoln and Angela Davis, as well 
  as his pride in the efficiency of the Officers' actions. 

\end{adjustwidth}


\clearpage

\fullcite{coming-together}

\begin{adjustwidth}{2.5ex}{0pt}

  Frederic Rzewski was a very unique musical voice. Much of his music 
  (as emphasized in David Metzer's ``Prisoner's Voices") is politically 
  charged, contemporary, and specifically related to anti-war and 
  prison abolition movements. Coming Together specifically is a piece 
  written in the wake of the Attica Prison riot, with text from one 
  of Melville's letters. Rzewski noted that he was impressed by 
  ``poetic quality of the text and by its cryptic irony". Encoded 
  in the text and in the musical construction is the image of Attica: 
  a bleak, rigid structure that restrains a Revolutionary Black Soul. 
  ``Coming Together", along with partner piece ``Attica" functions as 
  a primary impetus for this project. They will both function as major 
  creative products from 1970's post-Riot in my synthesis. 

\end{adjustwidth}

\clearpage

\fullcite{attica-poems}

\begin{adjustwidth}{2.5ex}{0pt}

  ``When the Smoke Cleared" is a marvelous collection. This 
  collection of poems was written in the years following the 
  Attica Riots by inmates still residing at Attica State. The 
  collection was compiled from works written in editor Celes 
  Tisdale's poetry writing workshops. Tisdale served as teacher 
  at Attica from 1972 to 1975, where he ran three 16 week poetry 
  workshops, as well as other classes. Of note is the emotional 
  friction and contrast between the writings of visitor Tisdale 
  and his students, who all seem to be warmer, more direct, and 
  more violent (though Tisdale uses the language 	``unpolished") 
  in their approaches to poetry and writing than Tisdale. The 
  primary poems I wish to analyse and use in my cultural synthesis 
  are: 
  \begin{itemize}
  
    \item Poet - Raymond X. Webster
    
    \item Black Dolphin and Haiku - Harold E. Packwood  
    
    \item What Makes a Man Free? - Clarence Phillips
    
    \item The Cure - ``Jamail" Robert Simms
    
    \item 1st Page - Daniel Brown
    
    \item Remember This - Celes Tisdale
  
  \end{itemize}
  
\end{adjustwidth}



\clearpage

\fullcite{blood-in-water}

\begin{adjustwidth}{2.5ex}{0pt}

  Heather Ann Thompson's Blood in the Water is simply what the 
  Attica Report could have been were it written to inform, not to 
  propagandize - and also if it were written 45 years later. Blood 
  in the Water is a comprehensive report on the when, where, how, what, 
  and, most importantly, \textit{why} of the events that occured at 
  Attica during the riots. The Attica Report starts with an encounter 
  between inmate Dewer and Officers Maronie and Curtiss. Though 
  the encounter is equally described in both the official Report and 
  in Blood in the Water, Thompson's recording focuses primarily on 
  the voice of the prisoners and the dissenting officers. By the 
  publication date of Blood in the Water, it is well understood 
  that the brutality of incarceration is obscured in the public 
  eye, and that the official reports on Attica (as assured by 
  Joshua Melville in Attica's Ashes) was in many ways 
  intentionally inconclusive - particularly in the racial makeup of the 
  different factions in the Attica riots and the militance of several 
  ``celebrity" figures. I plan to use Blood in the Water as a way to 
  ``check and balance" the information and the records in the official 
  Attica Report, as well as to reinforce the idea of oppression of the 
  prison image. 

\end{adjustwidth}


\clearpage

\nocite{*}

\printbibliography

\end{document} 
